%        File: homework_9.tex
%     Created: Fri Mar 08 09:00 AM 2019 PST
% Last Change: Fri Mar 08 09:00 AM 2019 PST
%
\documentclass{article}
\usepackage{amsmath, amssymb}
\usepackage{enumitem}
\usepackage{siunitx}

\begin{document}
\begin{center}
  \textbf{Homework 9}\\
  \textbf{Due Thursday, March 14th}
\end{center}
Instructions: For this assignment you may use any of the phase plane programs I have given you to make plots. 
\vspace{0.5cm}
\hrule
\vspace{0.5cm}
\begin{enumerate}
  \item Applying systems of DE's to physics. Using physics we can produce the following system of DE's that model a hanging pendulum (of length 1 meter). Let $\theta = \theta(t)$ be the angle of the pendulum in time, where $\theta = 0$ corresponds to a pendulum at rest. Let $v = \theta'(t)$ be the angular velocity of the pendulum. We can think of $\theta$ and $v$ as state variables, which leads to the system of equations
    \begin{align*}
      \frac{d\theta}{dt} &= v\\
      \frac{dv}{dt} &= - 9.8 \sin(\theta).
    \end{align*}
    (You do not need to know how these equations are derived, but if you are curious, I would be happy to show you!)
    \begin{enumerate}
      \item Using your favorite phase plane plotter, make a phase plane for the range $-\pi \leq \theta \leq \pi$ and $-5 \leq v \leq 5$. (No matter which plotter you use, you should tweak the settings to get rid of ``numerical failures.'') Print out your phase plane with three trajectories beginning at $(\theta, v) = (0, 1)$, $(-3, 0)$, and $(-3, 2)$. 
      \item For each of the three trajectories you made, describe the behaviour of the pendulum. 
      \item Add nullclines to your plot. Are there any equilibria? If so, what do they mean for the pendulum?
    \end{enumerate}
  \item A pendulum with drag. In problem 1, the equations you see ignore any drag force (either from friction of the pendulum or air resistance). We fix this by adding in a term into $\frac{dv}{dt}$ that makes the pendulum slow down more quickly the faster it moves. This now gives us the system
    \begin{align*}
      \frac{d\theta}{dt} &= v\\
      \frac{dv}{dt} &= -9.8 \sin(\theta) - 2v.
    \end{align*}
    \begin{enumerate}
      \item Make a phase plane with your favorite plotter, and print it out. Include three trajectories with the same initial conditions as problem 1a. 
      \item For each trajectory, explain what the pendulum is doing.
      \item Why do the solutions now ``spiral?''
    \end{enumerate}

  \item Predator-Prey with logistic growth. Take the rabbits and foxes example from class, but now add in logistic growth, which shows up as an extra term proportional to $r^2$ in the $\frac{dr}{dt}$ equation:
    \begin{align*}
      \frac{dr}{dt} &= 4 r - 0.4 r^2 - 2 r f\\
      \frac{df}{dt} &= -3f + rf.
    \end{align*}
    \begin{enumerate}
      \item Find the nullclines of this system. How do they differ from the predator-prey model without logistic growth?
      \item 
	Make a phase plane for the region $0 \leq r \leq 7$ and $0\leq f \leq 5$. (Careful, not every plotter I gave you will be able to make these ranges.)
      \item Describe the behaviour of the two populations over time using trajectories in this phase plane.
  \end{enumerate}

  \item The Fitzhugh-Nagumo equations without current are
    \begin{align*}
      \frac{dV}{dt} &= - V(V-c)(V-1) - R \\
      \frac{dR}{dt} &= a V - b R.
    \end{align*}

	\begin{enumerate}
	  \item For the parameter values of $c=0.4$, $a = 1$, and $b= 2$, make a phase plane with several trajectories beginning at locations with voltage at $0.4$. Print it out.
	  \item Describe the dynamics of this neuron. 
	  \item Is this neuron easily excitable? Why or why not?
	\end{enumerate}

  \item The Fitzhugh-Nagumo equations with current are
    \begin{align*}
      \frac{dV}{dt} &= - V(V-c)(V-1) - R + I\\
      \frac{dR}{dt} &= a V - b R.
    \end{align*}
    where $I$ is the external current supplied to the neuron. 
    \begin{enumerate}
      \item For the parameter values $c = 0.2$, $I = 1.5$, $a = 0.7$, and $b = 0.21$, make a phase plane with a trajectory beginning at rest, which is $(V,R) = (0,0)$. Print it out.
	\item Describe the dynamics of this neuron from this trajectory.
	\item Is this neuron easily excitable? Why or why not?
    \end{enumerate}
\end{enumerate}
\end{document}



